%% Esempio per lo stile supsi
\documentclass[twoside]{supsistudent} 

% per settare noindent
\setlength{\parindent}{0pt}


% Crea un capitolo senza numerazione che pero` appare nell'indice %
\newcommand{\problemchapter}[1]{%
  \chapter*{#1}%
  \addcontentsline{toc}{chapter}{#1}%
\markboth{#1}{#1}
}

% Numerazione delle appendici secondo norma
\addto\appendix{
\renewcommand{\thesection}{\Alph{chapter}.\arabic{section}}
\renewcommand{\thesubsection}{\thesection.\arabic{subsection}}}

\setcounter{secnumdepth}{5} 	%per avere più livelli nei titoli
\setcounter{tocdepth}{5}		%per avere più livelli nell'indice


\titolo{Non c'\`e due senza quattro}
\studente{Bastiano Coimbra de La Coronilla y Azevedo \vspace{1em}\\ Antonio Coimbra de La Coronilla y Azevedo }
\relatore{Donna Olimpia Chavez di Altamirano}
\correlatore{Nadinho}
\committente{Tango}
\corso{C1122 Sosia e Affini}
\modulo{M1424 Similarit\`a}
\anno{2012}



\begin{document}

\pagenumbering{alph}
\maketitle
\onehalfspacing
\frontmatter


\pagenumbering{roman}
\tableofcontents
\listoffigures					% Opzionale
\listoftables					% Opzionale

\newpage
\mainmatter
\pagenumbering{arabic}
\setcounter{page}{1}

\chapter{Titolazione}

\lipsum[13]

\section{Sezione}

\lipsum[23]
Esempio di citazione \cite{4538384}, \cite{5357331,4523385}, \cite{1705631}.
\footnote{Questa \`e una nota a pi\'e di pagina.}
\footnote{Questa \`e un'altra nota a pi\'e di pagina.}
\lipsum[23]

\subsection{Sotto sezione}

\texttt{Questo testo ha una spaziatura fissa}

\textit{Questo testo \`e in italico}

\textbf{Questo testo \`e in grassetto}

\textsc{Questo testo \`e in maiuscoletto}

\underline{Questo testo \`e sottolineato} \\

Citazione:
\begin{quote}
\lipsum[23]
\end{quote}

\chapter{Titolazione}

\lipsum[13]

\begin{itemize}
  \item Elemento A
  \item Elemento B
  \item Elemento C
\end{itemize}

%\begin{itemize}
%  \item[-] Elemento A
%  \item[-] Elemento B
%  \item[-] Elemento C
%\end{itemize}
%
%\begin{enumerate}
%  \item Alpha
%  \item Beta
%  \item Gamma
%\end{enumerate}

\lipsum[23]
%\section{Sezione}
%
%\lipsum[23]
%
%\subsection{Sotto sezione}
%
%Un po' di matematica: \newline
%
%\begin{math}
%\frac{n!}{k!(n-k)!} = {n \choose k}
%\end{math} \newline
%
%Un po' di matematica centrata:
%
%\begin{center}
%\begin{math}
%\frac{n!}{k!(n-k)!} = {n \choose k}
%\end{math}
%\end{center}
%
%Oppure con \$\$
%
%$$
%\frac{n!}{k!(n-k)!} = {n \choose k}
%$$
%
%Oppure anche direttamente nel testo ${1}\over{n}$ \\
%
%\lipsum[23]

\bibliographystyle{unsrt}
\bibliography{bibliografia}
\end{document}
